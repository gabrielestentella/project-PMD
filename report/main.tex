\documentclass[12pt]{article}
\usepackage{geometry}
\geometry{a4paper}
\usepackage{graphicx}
\usepackage{amssymb}
\usepackage{booktabs}
\usepackage[T1]{fontenc}


\title{Progetto di laboratorio}
\subtitle{Corso di Progettazione Model Driven del Software}
\author{Gabriele Stentella}
\date{}

\begin{document}
\maketitle
\section{Descrizione del problema}

% Questa sezione deve essere una versione sintetica del documento RASD (Requirements Analysis and Specification Document).
% In particolare, deve contenere, come minimo:
% \begin{enumerate}
% \item la descrizione e specifica dei requisiti distinguendo tra requisiti funzionali e non funzionali;\footnote{Va definito e usato in modo consistente un formato per la descrizione (e indicizzazione) dettagliata dei requisiti.}
% \item il glossario (con i termini in ordine alfabetico).
% \end{enumerate}

\subsection{Analisi e specifica dei requisiti}
Viene richiesto di realizzare un applicativo per la gestione dei tirocini che rispetti i seguenti requisiti:
\begin{itemize}
   \item Requisiti funzionali
   \begin{enumerate}
      \item Lo studente deve poter compilare una richiesta di attivazione del tirocinio che pu� essere interno o esterno
      \item Lo studente deve poter indicare i dati anagrafici relativi a se stesso e ai relatori, oltre alla data di inizio e di fine del tirocinio
      \item Il relatore deve ricevere una notifica via mail quando uno studente compila una richiesta di tirocinio con lui
      \item Il relatore deve poter visualizzare i dati relativi a tutti i tirocini in cui � coinvolto
      \item Il relatore deve poter approvare o rifiutare una richiesta di tirocinio
      \item Il relatore deve poter modificare i dati relativi a un tirocinio gi� approvato
      \item Il relatore deve poter terminare un tirocinio
      \item La commissione tirocini deve poter visualizzare i dati relativi a tutti i tirocini approvati dai relatori
      \item La commissione tirocini deve poter approvare o rifiutare una richiesta di tirocinio
   \end{enumerate}

   \item Requisiti non funzionali
   \begin{enumerate}
      \item Lo studente deve poter attivare un tirocinio soltanto se ha gi� superato tutti gli \emph{esami obbligatori} dei primi due anni, oppure ha acquisito almeno 120 CFU e deve ancora superare al pi� un esame fra quelli obbligatori nei primi due anni
      \item Il relatore deve poter terminare un tirocinio solo se si � raggiunta o superata la data di fine del tirocinio
      \item Il tirocinio deve durare almeno 14 settimane e non pi� di 24
   \end{enumerate}
\end{itemize}

\subsection{Glossario}
\begin{itemize} %DA ORDINARE IN ORDINE ALFABETICO
   \item CFU: Crediti Formativi Universitari, unit� di misura dell�impegno richiesto in termini di attivit� di studio o di apprendimento, un credito corrisponde convenzionalmente a 25 ore di impegno\footnote{fonte: https://www.unimi.it/it/corsi/orientarsi-e-scegliere/il-sistema-universitario}.
   \item esami obbligatori: gli insegnamenti elencati nelle tabelle delle "Attivit� formative obbligatorie" contenute nel manifesto degli studi reperibile a https://apps.unimi.it/files/manifesti/ita_manifesto_F68of4_2024.pdf.
   \item 
\end{itemize}

\section{Progettazione del Sistema}
\subsection{Diagramma dei casi d'uso}

Per ogni caso d'uso ci deve essere almeno una descrizione testuale del caso d'uso. Per i casi d'uso principali (almeno per i casi d'uso base) deve esserci anche la descrizione di un possibile scenario, eventualmente con l'indicazione delle possibili varianti.

\subsection{Descrizione degli scenari}

\begin{table}[htbp]
   \centering
    \begin{tabular}{|l|l|}
    \hline
    \textbf{Nome} & Nome del caso d'uso \\    \hline
    \textbf{Scopo} & Breve descrizione dello scopo del caso d'uso  \\  \hline
     \textbf{Attore} & Elenco attore/i coinvolti  \\  \hline
     \textbf{Pre-condizioni} &  \\ \hline
     \textbf{Trigger} &  \\  \hline
     \textbf{Descrizione sequenza eventi} &  \\  \hline
     \textbf{Alternativa/e} &  \\  \hline
     \textbf{Post-condizioni} &  \\    \hline
    \end{tabular}
\end{table}   
   
Per cambiare l'aspetto della tabella potete far riferimento a un qualsiasi manuale LaTeX o alla Tabella \ref{tab:tabexample} di esempio.

% Requires the booktabs if the memoir class is not being used
\begin{table}[htbp]
   \centering
   \caption{Table captions are better up top} 
   \begin{tabular}{@{} lcr @{}} % Column formatting, @{} suppresses leading/trailing space
      \toprule
      \multicolumn{2}{c}{Item} \\
      \cmidrule(r){1-2} % Partial rule. (r) trims the line a little bit on the right; (l) & (lr) also possible
      Animal    & Description & Price (\$)\\
      \midrule
      Gnat      & per gram & 13.65 \\
                & each     &  0.01 \\
      Gnu       & stuffed  & 92.50 \\
      Emu       & stuffed  & 33.33 \\
      Armadillo & frozen   &  8.99 \\
     \bottomrule
   \end{tabular}
   %\caption{Remember, \emph{never} use vertical lines in tables.}
   \label{tab:tabexample}
\end{table}

\subsection{Diagramma delle classi}

In questo stadio ha senso che si riporti e si discuta il diagramma delle classi di progetto. Il diagramma delle classi di programma verr� presentato e discusso nella sezione relativa all'implementazione.

Si ricordi che comunque il diagramma riportato in questa sezione:
\begin{enumerate}
\item Deve essere tracciabile con tutte le alte viste.
\item Deve essere tracciabile con il codice nel caso si decida di riportare un unico diagramma delle classi (quindi solo quello di programma).
\item Verr� valutato guardando se i principi chiave della progettazione OO siano stati considerati (classi altamente coese, poco accoppiamento tra classi, uso di gerarchia tra classi, ecc.) e considerando le associazioni tra classi.
\end{enumerate}


\subsection{Diagrammi di sequenza}

Esempio di come inserire una fotografia (con didascalia): 
\begin{figure}[htbp]
   \centering
   \includegraphics{images/minerva_piccola.png} % requires the graphicx package
   \caption{Example caption}
   \label{fig:example}
\end{figure}

\subsection{Diagrammi delle attivit\`a}

\subsection{Macchine di stato}

La notazione utilizzata per esprimere le macchine si stato deve essere usata in modo corretto, ad esempio i call event devono essere tracciabili con i messaggi dei casi d?uso e i metodi delle classi. Lo stesso si applica per tutte azioni che sono chiamate di metodi. 

\end{document}  